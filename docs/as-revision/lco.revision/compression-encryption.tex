\section{Compression, Hashing and Encryption}

\begin{questions}
\subsection{Compression}
	\question Contrast lossy and lossless forms of compression.
	\question Describe run length encoding and state whether it is lossy or lossless. 
	\question Is the storing of analogue data in digital form compression? How might such data be compressed? 
	\question Give two uses of
	\begin{parts}
		\part lossy compression 
		\part lossless compression
	\end{parts}
	\question Give three reasons why compression might be used. 
	\question Video when streamed may start being pixellated and then growing progressively smoother looking. Explain why this is the case. 
	\question What is a CODEC?

\subsection{Hashing}   

	\question What is the purpose of a hashing algorithm? 
	\question With the context of hashing, define: 
	\begin{parts}
		\part collision
		\part clustering
		\part load factor 
	\end{parts}
	\question Describe two alternative methods for dealing with collisions. 

\subsection{Encryption}
	\question Why might data be encrypted?   
	\question Define: 
	\begin{parts}
		\part key 
		\part plaintext 
		\part Caesar cipher 
		\part frequency analysis 
		\part one-time pad
		\part symmetric and asymmetric encryption  
		\part computational security 
	\end{parts} 
	\question Modulo arithmetic 
	\begin{parts}
		\part What is $21 + 15$ mod $26$?
		\part Today is Friday, what day will it be in 506 days' time? What was it 212 days ago? 
		\part If you head due East, turn through $130\deg$, what is your new bearing? 
	\end{parts}
	\question For every integer $b$ and positive integer $m$ there is exactly one integer $q$ and exactly one positive integer $r$ such that \[b = q\times m + r\], the \emph{quotient and remainder theorem}. Find $r$ and $q$ when 
	\begin{parts}
		\part $b=37, m=12$
		\part $b=76, m=60$
		\part $b=-37,m=12$
	\end{parts}

	\question Encipher a short message (< 50 characters) using a Caesar cipher. Swap with a colleague. Attempt to decipher their message. 
	\question Explain why a Vernam cipher is considered unbreakable. What conditions must be observed? 
	\question How would you set about breaking a non-trivial cipher? 
	\question How would public key crypto be used to 
	\begin{parts}
		\part encrypt 
		\part authenticate 
	\end{parts}
	messages? 
\end{questions}